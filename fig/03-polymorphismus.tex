\documentclass{lecturefig}
\usetikzlibrary{trees}
\begin{document}
\begin{frame}[fragile]
  \begin{tikzpicture}[
    code/.style={fill=codecolor,font=\ttfamily\footnotesize,align=left},
    level 1/.style={sibling distance=6.5cm},
    level 2/.style={sibling distance=3cm},
    lecture/.style={
      label={[visible on=<2->, lecture circle,inner sep=1pt,anchor=mid]north west:\scriptsize #1},
    },
    ]
    \tikzset{every child/.style={execute at begin node={\strut}}}
    \node[]
       {Polymorphismus}
       child { node (universal) {universell}
         child {
           node (inclusion) {Inklusion}
         }
         child {
           node[] (parametric) {Parametrisch}
         }
       }
       child { node (ad-hoc) {ad-hoc}
         child {
           node[] (overloading) {Überladung}
         }
         child {
           node[] (coercion) {Coercion}
         }
       };

       \node[code,below=.1 of inclusion,align=left,lecture=3] (inclusion-code) {
         class Derived\\
         \hspace{2ex}: public Base
       };

       \node[code,below=.1 of parametric,align=left,lecture=3] (parametric-code) {
         uint32\_t\\length(List[N])
       };

       \node[code,below=.1 of overloading,align=left,lecture=4] (overloading-code) {
         do(Point2D p);\\
         do(Point3D p);
       };

       \node[code,below=.1 of coercion,align=left,lecture=5] (coercion-code) {
         3 + 4.0
       };

       \node[fit=(universal)   (inclusion-code) (parametric-code), draw, thick,srared, visible on=<2>]{};
       \node[fit=(overloading) (overloading-code), draw,thick,srared, visible on=<{3,5}>] (hl-overloading) {};
       \node[fit=(coercion)    (coercion-code),    draw,thick,srared, visible on=<4>] {};
       \node[fit=(inclusion)   (inclusion-code),   draw,thick,srared, visible on=<5>] (hl-inclusion) {};

       \only<5>{
         \draw[<->,thick,srared] (hl-overloading.north) -- ++(up:1em) -| node[above,pos=0.2]{Interaktion}(hl-inclusion.north);
       }


  \end{tikzpicture}
\end{frame}
\end{document}
