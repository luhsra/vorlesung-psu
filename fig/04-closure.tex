\documentclass{lecturefig}
\usepackage{fig/04-common}
\begin{document}
\begin{frame}[fragile]
  \begin{tikzpicture}[
    call frame/.style={
      draw,fill=white,
      rectangle split,
      rectangle split parts=2,
      align=left, font=\ttfamily
    },
    remember picture,
    ]

    \node[call frame,text width=3.5cm,visible on=<2->] (f1) {
      outer(10)
      \nodepart{two}
      init = 10\\
      \alert<5->{x = \only<2-5>{10}\only<6>{11}\only<7>{12}\only<8->{13}}\\
      \tikzmark{inner}inner = \alt<2>{?}{<closure>}
    };

    \begin{scope}[visible on=<3->]
      \node[call frame,below=1 of f1] (closure) {
        <closure>
        \nodepart{two}
        caller = <frame>\tikzmark{closure}
      };
      \draw[->] (pic cs:inner)++(up:0.2\baselineskip)--++(west:1em) |- (closure.text west);
      \draw[->] (pic cs:closure)++(up:0.2\baselineskip)--++(east:2em) |- (f1.text east);
    \end{scope}

    \begin{scope}[visible on=<4->]
      \draw[<-] (closure) -- ++(down:1cm) node[draw,below]{global variable: f};
    \end{scope}


    \node[left=1.3 of f1.north west, anchor=north east,  align=left,fill=codecolor,font=\ttfamily] (code) {
      function outer(init) \{\\
      \ \ \tikznode[name=def,fill=letcolor]{var x} = init;\\
      \ \ function inner() \{\\
      \ \ \ \ \tikznode[name=use,fill=refcolor]{x} = x + 1;\\
      \ \ \ \ return x;\\
      \ \ \}\\
      \ \ return inner;\tikznode[name=ret]{}\\
      \}
    };

    \draw[<-] (def.west) -- ++(west:1em) |- (use);

    \draw[<-,srared,thick,visible on=<2>] (def-|code.east) -- ++(east:1em);
    \draw[<-,srared,thick,visible on=<3>] (ret-|code.east) -- ++(east:1em);
    \draw[<-,srared,thick,visible on=<5-8>] (use-|code.east) -- ++(east:1em);


    \begin{scope}[visible on=<2->]
        \node[width as=code,below=0.2 of code,draw,minimum height=4.3\baselineskip] (console) {};
        \node[anchor=north west,at=(console.north west),font=\ttfamily,align=left] {
          \strut> var f = outer(10);\\
          \strut\only<4->{<closure>\\>}\only<5->{[f(), f(), f()]\\}
          \only<9->{[11, 12, 13]}
        };
      \end{scope}
    \end{tikzpicture}
\end{frame}
\end{document}
